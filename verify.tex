\documentclass[UTF8,linespread=1.236]{ctexart}
\usepackage{subfig}
\usepackage{listings}
\usepackage{amsmath}
\usepackage{amssymb}
\usepackage{booktabs}
\usepackage{float}
\usepackage[Q=yes]{examplep}
\usepackage[a4paper,centering,top=3cm,left=2.6cm]{geometry}
\usepackage{pgf}
\usepackage{tikz}
\usetikzlibrary{arrows,automata,positioning}

\pagestyle{plain}
\ctexset {
    section = {
        name = {},
        number = {},
        aftername = {},
    },
    subsection = {
        name = {,、},
        number = \chinese{subsection},
        aftername = {},
    },
    subsubsection = {
        name = {},
        number = \arabic{subsubsection},
    }
}
\newcommand\cu[1]{\boldsymbol{#1}}
\newcommand\pypx[2]{{{\partial {#1}} \over {\partial {#2}}}}
\newcommand\pie{{}^\prime}
\newcommand\nextsign{{}^\ast}
\begin{document}

\title{《电磁场与波B》课程设计}
\author{电子科学与工程学院\ \ 傅宣登 (2016030102010)}

\maketitle
%\thispagestyle{empty}

\section{关于均匀平面波与圆极化波能否同时存在的探讨}

\subsection{均匀平面波}

\subsubsection{一般波动方程}

对于电容率为 $\varepsilon$,磁导率为 $\mu$,电导率为 $\sigma$
的无源均匀媒质,
麦克斯韦方程是
\begin{equation}%\label{eq81}
    \nabla \times \cu{E} = - \pypx{\cu{B}}{t}
\end{equation}
\begin{equation}
    %\label{eq82}
    \nabla \times \cu{H} = \cu{J} + \pypx{\cu{D}}{t}
\end{equation}
\begin{equation}
    %\label{eq83}
    \nabla \cdot \cu{B} = 0
\end{equation}
\begin{equation}
    %\label{eq84}
    \nabla \cdot \cu{D} = \rho
\end{equation}
其中 $\cu{J} = \sigma\cu{E}$,$\cu{B} = \mu\cu{H}$,
$\cu{D} = \varepsilon\cu{E}$,
$\rho = 0$. 于是有
\begin{equation}
    \label{eq81}
    \nabla \times \cu{E} = -\mu \pypx{\cu{H}}{t}
\end{equation}
\begin{equation}
    \label{eq82}
    \nabla \times \cu{H} = \sigma\cu{E} + \varepsilon\pypx{\cu{E}}{t}
\end{equation}
\begin{equation}
    \label{eq83}
    \nabla \cdot \cu{H} = 0
\end{equation}
\begin{equation}
    \label{eq84}
    \nabla \cdot \cu{E} = 0
\end{equation}


\subsubsection{介质中的平面波}

\subsection{圆极化波}

\subsubsection{极化的概念}

\subsubsection{圆极化}

\subsection{同时满足两种性质}


%\clearpage

\end{document}
